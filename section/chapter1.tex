\chapter{Pengenalan Keamanan Sistem Informasi}
Dunia Keamanan sistem informasi melingkupi aspek aspek yang harus diperhatikan dalam hal asset - asset yang perlu dilindungi. Contoh asset yang dimaksud seperti perangkat keras, perangkat lunak, database, informasi penting serta beberapa fasilitas lain.
Menurut ISO/IEC 27002, 2005 pada dasarnya yang ingin dilindungi adalah berkaitan dengan lima komponen dasar sistem informasi yaitu perangkat keras, perangkat lunak, pengguna, data dan prosedur. Berikut empat karakteristik dasar yang dapat diketahui apabila ingin menerapkan solusi pengamanan sistem informasi :
\begin{enumerate}
\item
Harus memiliki sebuah sistem komputerasi yang layak dilindungi seperti personal komputer, jaringan  (\textit{local area}) atau jaringan yang lebih luas lagi. Dengan kata lain fasilitas komputer dan jaringan data mutlak diperlukan. 
\item
Memiliki sebuah divisi/bagian teknologi informasi yang menangani berbagai kegiatan penunjang untuk berbagai aplikasi bisnis.
\item
Mempunyai data, informasi dan sistem jaringan yang berharga dan layak untuk dijaga, dan dapat menyebabkan kerugian besar apabila data, informasi tersebut diketahui orang banyak.
\item
Belum memiliki kebijakan tata kelola teknologi informasi terutama yang berkaitan dengan kebijakan pengelolaan keamanan sistem informasi, atau sudah memiliki tapi belum mengacu pada standariasi (ISO).
\end{enumerate}
Tugas dengan cara dikumpulkan dengan pull request ke github dengan menggunakan latex pada repo yang dibuat oleh asisten IRC.

\section{Kejahatan Komputer}
Kejahatan komputer yang paling sering terjadi adalah pencurian kartu kredit, pembobolan dana nasabah dan dunia perbankan sempat dihebohkan dengan kegiatan \textit {skimmer}. Kejahatan perbankan sangat marak karena kebanyakan \textit{hacker} melakukan hal tersebut karena uang dan sumber uang terbesar di dunia adalah bank. Tingkat keamanan bank atau bentuk organisasi keuanganlah  yang selalu menjadi sasaran peretasan. Karena tingginya keamanan perbankan, maka sasaran para \textit{hacker} adalah sisi nasabah. Nasabah merupakan rantai transaksi yang paling lemah. Menurut Thomas porter ada beberapa istilah dalam dunia kejahatan komputer sebagai berikut :
\begin{enumerate}
\item
\textit{Computer Abuse} yaitu tindakan sengaja dengan melibatkan satu pelakukejahatan atau lebih sehingga dapat memperoleh keuntungan bagi pelaku dan kerugian bagi korban.
\item
\textit{Computer Crime} yaitu tindakan melanggar hukum yang membutuhkan banyak pengetahuan tentang komputeragar pelaksanaannya dapat berjalan dengan baik.
\item
\textit{Computer realter crime} yaitu kejahatan yang berkaitan dengan komputer yang tidak terbatas pada kejahatan bisnis kerah putih (\textit{white collar crime})

\end{enumerate}

\section{Klasifikasi Kejahatan Komputer}
Pada umumnya kejahatan komputer dapat diklasifikasikan kedalam empat tipe berdasarkan lubang keamanan (\textit{Vulnerability}). Berikut empat tipe keamanan komputer berdasarkan lubang keamanannya menurut david icove \cite{Icove} :  

\begin{enumerate}
\item
Keamanan yang bersifat fisik (\textbf{\emph{physical security}})
\\Termasuk akses orang ke gedung, peralatan atau media yang digunakan. Contoh kejahatan jenis ini adalah sebagai berikut :
\begin{description}
\item[a] Berkas-berkas dokumen yang telah dibuang ke tempat sampah yang mungkin memuat informasi password dan username.
\item[b] Pencurian komputer laptop.
\item[c] Serangan \textit {DDos Attack}
\item[d] Pemutusan jalur listrik sehingga tidak berfungsi secara fisik.
\end{description}
\item
Keamanan yang berhubungan dengan orang (\textbf{\emph{personal security}}). 
\\ Tipe keamanan jenis ini termasuk kepada identifikasi, profile risiko dari pekerja di sebuah perusahaan. Dalam dunia keamanan informasi salah satu faktor terlemah adalah dari tipe jenis ini. Kejahatan jenis ini sering menggunakan metode yang disebut \textit {social engineering}.
\item
Keamanan dari data dan media serta teknik komunikasi (\textbf{\emph{Communication security}}).
\\ Tipe keamanan jenis ini banyak menggunakan kelemahan yang ada pada perangkat lunak, baik perangkat lunak aplikasi atapun perangkat lunak yang digunakan dalam mengelola database.
\item
Keamanan dalam opeasi  \textbf{\emph{Management security}}).
\\Kebijakan digunakan untuk mengelolasistem keamanan , prosedur sebelum maupun setelah serangan terjadi , mempelajari manajemen resiko seperti dampak sebab dan akibat dari sebuah serangan.

\end{enumerate}


\section{Macam Pengelompokan Hacker}
Ada Beberapa jenis hacker yang dapat dikumpulkan. Berikut ini ada empat tipe hacker yang dilihat dari sisimotivasi dan kegiatan mereka :
\begin{enumerate}
	\item
 \textbf{\emph{Black hat}}, Kumpulan individu yang memiliki keahlian tinggi dibidang keamanan komputer yang memiliki motivasi melakukan tindakan \textit{destructif} terhadap sistem komputer tertentu, sasarannya adalah sebuah imbalan. \textit{Hacker} jenis ini yang lebih sering disebut dengan \textit{crackers}
	\item
\textbf{\emph{White hat}}, Sejumlah individu profesional yang memiliki keahlian di bidang internet dan sistem komputer yang bertugas untuk menjaga keamanan sebuah fasilitas sistem internet dari serangan pihat tertentu yang merugikan. \textit{Hacker} jenis ini yang lebih sering disebut dengan \textit{security analysts}
\item
\textbf{\emph{White hat}}, Sekumpulan individu yang kadang-kadang melakukan suatu tindakan offensive dan kadang melakukan tindakan defensive untuk tujuan tertentu terkait keamanan sebuah sistem komputer.
\item 
\textbf{\emph{Suicide Hacker}},
\end{enumerate}

